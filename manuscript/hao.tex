\documentclass{article}
\usepackage{ee477}
\usepackage{mathrsfs}
\usepackage{amssymb}
\usepackage{graphicx}
\usepackage[margin=1in]{geometry} 
\usepackage{amsmath,amsthm,amssymb}
\usepackage{enumitem}
\usepackage{amsmath}

\newcommand{\indep}{\rotatebox[origin=c]{90}{$\models$}}

\begin{document}
 
\title{PSB-GNN}
\maketitle

\section{Introduction} 
Evaluating the toxicity of chemicals is essential to pharmaceutical research.
The task has traditionally been conducted through in vivo models \cite{fischer2020toxicity}.
However, in vivo testing bears major limitations as the number of chemicals in need of assessment explodes.
Meanwhile, it remains challenging to extrapolate findings from model organisms to humans.
In the last decade, in silico testing has become an alternative approach to analyze toxicity profiles of chemicals \cite{raies2016silico}.
The development of toxicity prediction models is facilitated by high-throughput screening resources such as Tox21 \cite{huang2016quantitative}, which provides thousands of data points for model traninig.
A common in silico approach is quantitative structure-activity relationship (QSAR) modeling \cite{cherkasov2014qsar}.
QSAR models first quantify chemical structure into into molecular descriptors or fingerprints, then relate them to the endpoints by classification algorithms.
The accuracy of QSAR models in toxicity prediction largely depend on endpoint of interest \cite{valerio2009silico}.
Despite of achieving decent performance for some endpoints, QSAR models often failed to explain the cellular mechanisms underlying structure-toxicity associations \cite{polishchuk2017interpretation}.
The lack of interpretability arises from difficulties in linking structure properties of high relevance to toxicity genes and pathways \cite{cruz2017systemic}. 

\section{Method}

\subsection{Tox21 datasets}
We obtained active and inactive compounds of 52 toxicity assays from Tox21 data portal (\url{https://tripod.nih.gov/tox21/assays/}) \cite{huang2016quantitative}. 
Compounds were uniquely identified by their PubChem IDs. 
We removed compounds with inconvlusive or ambiguous screening results.  

\subsection{Baseline QSAR models}
We built QSAR models to predict the response of compounds to each toxicity assay.
To quantify structure properties of compounds, we adpoted 166 MACCS fingerprints that cover most interesting chemical features for drug discovery. 
We used "rcdk" package \cite{guha2007chemical} to compute MACCS fingerprints of chemicals from their SMILES strings.  
Two classification methods were implemented for the task: random forest and gradient boosting. 
Each model was trained with 80 percent of all available instances, then evaluted with the remaining 20 percent.
We tuned six hyperparameters for each random forest model, and five hyperparameters for each gradient boosting model (\textbf{Table S1}). 
We used grid search to find the optimal hyperparamter setting that minimizes binary cross entropy loss on training data.  
The optimal performance for all datasets can be found in \textbf{Table S2}. 


\bibliographystyle{unsrt}
\bibliography{psb-gnn}
 
\end{document}