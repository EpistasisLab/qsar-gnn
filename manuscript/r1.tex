\documentclass{ibilttr}
\usepackage[utf8]{inputenc}
\usepackage[english]{babel}

\usepackage{csquotes}
\renewcommand{\mkbegdispquote}[2]{\itshape}

% These options should be set by the user:
\fromname{Joseph~D.~Romano}
\fromdegrees{PhD}
\fromtitle{Postdoctoral Researcher}
\fromdept{Department of Biostatistics,\\[-1pt]Epidemiology, and Informatics}
\toname{PSB 2022 Organizing Committee}
%\sigfile{./img/romanosig}


\begin{document}
Please find in this letter the responses to reviewer comments for our
recently accepted paper titled \textit{Improving QSAR Modeling for
Predictive Toxicology using Publicly Aggregated Semantic Graph
Data}. We would like to thank the conference organizers, track
organizers, and especially the reviewers for their thoughtful and
detailed responses to our paper.

Below, we have reproduced each of the reviewers' comments and
concerns along with narrative responses to each of them. Although
there are no substantial changes made to the text of the manuscript,
we do feel that the tweaks suggested by reviewers have significantly
improved the paper by closing some remaining gaps in the analysis and
providing more complete background such that it will be more useful
for a wider audience of readers. Additionally, please note that some
of the edits were made in the Supplementary Materials rather than the
main text, which was necessitated by the 12-page limit of the main
manuscript.

\textbf{\sffamily Reesponses to reviewer comments:}

\textit{\textbf{Reviewer 1}}

\vspace{-1em}
\begin{displayquote}
The labelling on Figure 3 was slightly confusing but perhaps I'm
accustomed to different formatting in my discipline.
\end{displayquote}

We have tweaked the labeling of the violin plot (and the caption) to reduce
the overall amount of text and make the important parts of the plot
more intuitive and visually obvious.

\begin{displayquote}
\ldots I wonder if additional, more observational, data could be
intgegrated into the graph data and ultimately the GNN? The
availability of such data may be sparse but I was wondering if
additional data from epidemiological or clinical studies would be
useful?
\end{displayquote}

This is indeed an insightful and important comment, and an area we are
currently investigating as the subject of a follow-up
study. Specifically, we are looking at harmonizing chemicals in the
graph database with drug and environmental chemical exposures
documented in electronic health record data, and seeing if the EHR
data corroborates novel findings from the GNN approach. Although it is
not immediately clear how, we are also exploring methods for including
observational data directly in the GNN (e.g., each patient will be
their own node in the graph, and we will use clinical findings,
genetic data, and exposures to link patients to corresponding entities
in the graph). We have added some brief allusions to this proposed
future work in the Discussion section of the paper.

\textit{\textbf{Reviewer 2}}

\vspace{-1em}
\begin{displayquote}
I recognize the page limitations, but the paper could be strengthened
with a bit more detail about the algorithm used in the body of the
work.
\end{displayquote}

Thank you for this comment. We did indeed hope to include more details
on the model/algorithm but ended up having to keep it minimal to fit
the 12-page limit. However, we did remove several extraneous sentences
that allowed us to provide better detail as requested. Please refer to
the Appendix to see the changes.

\begin{displayquote}
I do think that while the improvement is statistically significant,
I'd like to know if the improvement in algorithm performance is the
equivalent of ``clinically relevent'' to drug development.
\end{displayquote}

Although this is definitely the long-term goal of our research, we
were careful to avoid making claims that the outcomes are clinically
significant. However, we have added additional text to the Discussion
explaining how our results can be considered \textit{pre-clinically}
significant in that they can directly guide the drug discovery
industry in prioritizing compounds based on their predicted safety
profiles.

\begin{displayquote}
Recognizing page limitations, the paper would benefit from a brief
description of other problems that are being tackled with GCNs\ldots
\end{displayquote}

Thank you for understanding our limitations in adding much additional
text. However, this is an important comment, and we have been able to
add a sentence that briefly introduces existing applications of GCNs.

\begin{displayquote}
There is ample literature on gene networks, but I'd like to see a
theoretical discussion of GCNs in the context of gene networks as
well, because toxicity is a very challenging problem and a solid
foundation in theory would help move the field forward.
\end{displayquote}

\textit{\textbf{Reviewer 3}}

\vspace{-1em}
\begin{displayquote}
Presentation of some parts (e.g., the first sentence of the abstract)
can be improved more, but it is very minor. Algorithm can be more
detailed, but I understand due to the limited pages.
\end{displayquote}

\begin{displayquote}
In the experiment, GNN showed higher variance than the others,
and it was justified that it is because that neural networks tend to
struggle as data become more sparse. I am not quite convinced for
that. If data is imbalanced, cost function with class weights can be
considered, and weighted F-1 score can be considered.
\end{displayquote}

\begin{displayquote}
I am wondering the performance with conventional neural networks
because it compared only non-deep learning methods.
\end{displayquote}

\begin{displayquote}
I am wondering how to choose threshold to compute F-1 scores.
\end{displayquote}

\vspace{2ex} Sincerely,

\includegraphics[width=1.5in]{romanosig}\\[-0.2\baselineskip]

Joseph~D.~Romano, PhD\\ Postdoctoral Researcher

\end{document}


%%% Local Variables:
%%% mode: latex
%%% TeX-master: t
%%% End:
